\documentclass[a4paper]{article}
\usepackage[utf8]{inputenc}
\usepackage[T1]{fontenc}
\usepackage[french]{babel}
\usepackage[left=2cm,right=2cm,top=2cm,bottom=2cm]{geometry}


%\documentclass[french]{article} % Nous pourrons tester avec l’option titlepage
%\usepackage{babel}

\date{\today}
\author{\bsc{DUPUY-\--CORTI Clara} \and \bsc{PELTZER Léon}}
\title{Rapport du projet algorithmique et programmation}

\begin{document}

{\large
\maketitle      

Mettre un petit texte d'introduction avant le sommaire ... par exemple :
\\
Dans ce document, vous trouverez toute la documentation concernant le projet :
\begin{itemize}
\item les explications des différentes classes, méthodes, et fonctions définies
\item les problèmes rencontrés au cours de l'écriture du code
\item ...
\end{itemize}


\tableofcontents

\newpage

   \section*{Introduction}
   \addcontentsline{toc}{section}{Introduction}
      texte de l’introduction

   \section{Problèmes rencontrés}
      Cette première partie permettra d'expliquer les différents problèmes auxquels nous avons fait face et la partie suivante décrira les moyens utilisés pour résoudre les problèmes.
      \subsection{blabla}
    \begin{enumerate}
   
\item D'abord créer la grille avant les mouvements et les conditions de mort pas saturation des animaux pour 

\item Faut-il stocker à chaque étape les voisins d'un objet (animal) dans un dico ou liste pour la chasse et la mort par sur-population ?

\item Problème d'affichage on avait une boucle while true qui faisait planter le programme
\item Le lapin bouffe le renard

\item Le lapin ne bouge pas

\item pb quand un renard veut se mélanger dans les lapins pour les bouffer, ça crash (c'est pas ça le problème --> pb avec les lapins )

    \end{enumerate}

%      \subsection{blabla 2}
%         texte de la sous-section 2
%      \subsection{Sous-section 3}
%         texte de la sous-section 3


\newpage

   \section[Avancés]{Les résolutions des problèmes de la partie 1}
   
- on a une bonne base pour travailler \\ 
class fox de base et rabbit de base 
classe garssaland de base avec les minimum pour le projet 


\newpage

   \section{Objectifs}
      texte d’introduction de la deuxième section
      \subsection{Sous-section 1}

         \begin{itemize}
         	\item Peaufiner
         	\item Commenter
         	\item Améliorer 

         \end{itemize}
         
%      \subsection{Sous-section 2}
%         texte de la sous-section 2


\newpage
   
   \section*{Conclusion}
   \addcontentsline{toc}{section}{Conclusion}
      texte de la conclusion 
   
   
\newpage

   \appendix  % On passe aux annexes
   \section{Annexe 1}
       texte de l’annexe 1 


\newpage
  
   \section{Annexe 2}
      texte de l’annexe 2




}
\end{document} 



